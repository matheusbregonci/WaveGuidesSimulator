\documentclass[a4paper,12pt]{article}
\usepackage[utf8]{inputenc}
\usepackage[portuguese]{babel}
\usepackage{amsmath}
\usepackage{amsfonts}
\usepackage{amssymb}
\usepackage{graphicx}
\usepackage{geometry}
\usepackage{float}
\usepackage{enumerate}
\usepackage{hyperref}
\usepackage{array}
\usepackage{booktabs}

\geometry{margin=2.5cm}

\title{Reflexão e Transmissão em Guias de Onda Cilíndricas:\\
Teoria e Matriz de Espalhamento}
\author{Simulador de Guias de Onda}
\date{\today}

\begin{document}

\maketitle

\tableofcontents
\newpage

\section{Introdução}

Os guias de onda cilíndricos são estruturas fundamentais em sistemas de microondas e telecomunicações. Este documento apresenta a teoria completa para análise de reflexão e transmissão nestes guias, desenvolvendo a formulação matemática necessária para cálculo da matriz de espalhamento.

\section{Fundamentos Teóricos}

\subsection{Equações de Maxwell em Coordenadas Cilíndricas}

Para um guia cilíndrico de raio $a$, as equações de Maxwell em coordenadas cilíndricas ($\rho$, $\phi$, $z$) são:

\begin{align}
\nabla \times \vec{E} &= -j\omega\mu \vec{H} \\
\nabla \times \vec{H} &= j\omega\epsilon \vec{E}
\end{align}

onde $\omega = 2\pi f$ é a frequência angular, $\mu$ a permeabilidade e $\epsilon$ a permissividade do meio.

\subsection{Modos de Propagação}

Em guias cilíndricos, existem dois tipos fundamentais de modos:

\subsubsection{Modos Transversais Elétricos (TE)}
Para modos TE, $E_z = 0$ e o campo magnético longitudinal é dado por:

\begin{equation}
H_z(\rho,\phi,z) = A_{nm} J_n(k_c\rho) \cos(n\phi + \phi_0) e^{-j\beta z}
\end{equation}

onde:
\begin{itemize}
    \item $J_n$ é a função de Bessel de primeira espécie de ordem $n$
    \item $k_c$ é o número de onda de corte
    \item $\beta$ é a constante de propagação
    \item $A_{nm}$ é a amplitude do modo
\end{itemize}

\subsubsection{Modos Transversais Magnéticos (TM)}
Para modos TM, $H_z = 0$ e o campo elétrico longitudinal é:

\begin{equation}
E_z(\rho,\phi,z) = B_{nm} J_n(k_c\rho) \cos(n\phi + \phi_0) e^{-j\beta z}
\end{equation}

\subsection{Condições de Contorno e Frequências de Corte}

\subsubsection{Modos TE}
A condição de contorno na parede metálica ($\rho = a$) requer:
\begin{equation}
\frac{\partial J_n(k_c a)}{\partial \rho} = 0
\end{equation}

Isso resulta em:
\begin{equation}
k_{c,nm}^{(TE)} = \frac{p'_{nm}}{a}
\end{equation}

onde $p'_{nm}$ são os zeros da derivada da função de Bessel $J'_n(p'_{nm}) = 0$.

\subsubsection{Modos TM}
Para modos TM, a condição de contorno é:
\begin{equation}
J_n(k_c a) = 0
\end{equation}

Resultando em:
\begin{equation}
k_{c,nm}^{(TM)} = \frac{p_{nm}}{a}
\end{equation}

onde $p_{nm}$ são os zeros da função de Bessel $J_n(p_{nm}) = 0$.

\subsection{Frequências de Corte}
A frequência de corte para cada modo é:
\begin{equation}
f_c = \frac{k_c c}{2\pi\sqrt{\epsilon_r \mu_r}}
\end{equation}

onde $c$ é a velocidade da luz no vácuo.

\section{Propagação e Impedâncias}

\subsection{Constante de Propagação}
Para frequências acima do corte ($f > f_c$):
\begin{equation}
\beta = \sqrt{k^2 - k_c^2}
\end{equation}

onde $k = \omega\sqrt{\mu\epsilon}$ é o número de onda no meio.

\subsection{Impedância Característica}

\subsubsection{Modos TE}
\begin{equation}
Z_{TE} = \frac{\omega\mu}{\beta} = \frac{\eta}{\sqrt{1 - (f_c/f)^2}}
\end{equation}

\subsubsection{Modos TM}
\begin{equation}
Z_{TM} = \frac{\beta}{\omega\epsilon} = \eta\sqrt{1 - (f_c/f)^2}
\end{equation}

onde $\eta = \sqrt{\mu/\epsilon}$ é a impedância intrínseca do meio.

\section{Coeficientes de Reflexão e Transmissão}

\subsection{Descontinuidades em Guias Cilíndricos}

Quando uma onda encontra uma descontinuidade (mudança de raio, material, etc.), parte da energia é refletida e parte é transmitida.

\subsection{Coeficiente de Reflexão}
Para uma descontinuidade entre dois guias com impedâncias $Z_1$ e $Z_2$:

\begin{equation}
\Gamma = \frac{Z_2 - Z_1}{Z_2 + Z_1}
\end{equation}

\subsection{Coeficiente de Transmissão}
\begin{equation}
T = \frac{2Z_2}{Z_2 + Z_1} = 1 + \Gamma
\end{equation}

\section{Matriz de Espalhamento}

\subsection{Definição da Matriz S}
A matriz de espalhamento relaciona as ondas incidentes e refletidas:

\begin{equation}
\begin{bmatrix}
b_1 \\
b_2
\end{bmatrix} =
\begin{bmatrix}
S_{11} & S_{12} \\
S_{21} & S_{22}
\end{bmatrix}
\begin{bmatrix}
a_1 \\
a_2
\end{bmatrix}
\end{equation}

onde:
\begin{itemize}
    \item $a_i$: ondas incidentes normalizadas
    \item $b_i$: ondas refletidas/transmitidas normalizadas
    \item $S_{ij}$: parâmetros de espalhamento
\end{itemize}

\subsection{Interpretação Física}
\begin{align}
S_{11} &= \text{coeficiente de reflexão na porta 1} \\
S_{21} &= \text{coeficiente de transmissão da porta 1 para 2} \\
S_{12} &= \text{coeficiente de transmissão da porta 2 para 1} \\
S_{22} &= \text{coeficiente de reflexão na porta 2}
\end{align}

\subsection{Propriedades da Matriz S}

\subsubsection{Conservação de Energia}
Para um sistema sem perdas:
\begin{equation}
|S_{11}|^2 + |S_{21}|^2 = 1
\end{equation}
\begin{equation}
|S_{22}|^2 + |S_{12}|^2 = 1
\end{equation}

\subsubsection{Reciprocidade}
Para sistemas recíprocos:
\begin{equation}
S_{12} = S_{21}
\end{equation}

\section{Cálculo Prático dos Parâmetros S}

\subsection{Modelo de Cavidade Ressonante}
Para uma cavidade cilíndrica de comprimento $L$, os parâmetros S podem ser calculados considerando:

\subsubsection{S11 - Reflexão}
\begin{equation}
S_{11} = \frac{Z_{in} - Z_0}{Z_{in} + Z_0}
\end{equation}

onde $Z_{in}$ é a impedância de entrada da cavidade e $Z_0$ a impedância de referência.

\subsubsection{S21 - Transmissão}
Para uma cavidade com acoplamento fraco:
\begin{equation}
S_{21} = \frac{2\sqrt{Q_{e1}Q_{e2}}}{Q_L} \cdot \frac{1}{1 + 2jQ_L\frac{\Delta f}{f_0}}
\end{equation}

onde:
\begin{itemize}
    \item $Q_{e1}, Q_{e2}$: fatores de qualidade externos
    \item $Q_L$: fator de qualidade carregado
    \item $\Delta f = f - f_0$: desvio da frequência de ressonância
\end{itemize}

\subsection{Efeitos de Múltiplos Modos}
Em guias cilíndricos, múltiplos modos podem se propagar simultaneamente. O parâmetro S total é:

\begin{equation}
S_{total} = \sum_{n,m} S_{nm} \cdot \text{peso}_{nm}
\end{equation}

onde o peso depende da excitação e acoplamento de cada modo.

\section{Implementação Computacional}

\subsection{Algoritmo para Cálculo de S11}
\begin{enumerate}
    \item Calcular frequências de ressonância para todos os modos relevantes
    \item Para cada frequência de análise:
    \begin{enumerate}
        \item Identificar modos próximos à ressonância
        \item Calcular contribuição de cada modo
        \item Somar contribuições ponderadas
    \end{enumerate}
    \item Aplicar conservação de energia e reciprocidade
\end{enumerate}

\subsection{Algoritmo para Cálculo de S21}
\begin{enumerate}
    \item Usar modelo de ressonador acoplado
    \item Considerar perdas e acoplamento entre portas
    \item Garantir $|S_{11}|^2 + |S_{21}|^2 \leq 1$
\end{enumerate}

\section{Validação e Comparação}

\subsection{Casos Limites}
\begin{itemize}
    \item Guia infinito: $S_{11} = 0$, $S_{21} = e^{-j\beta L}$
    \item Curto-circuito: $S_{11} = -1$, $S_{21} = 0$
    \item Circuito aberto: $S_{11} = 1$, $S_{21} = 0$
\end{itemize}

\subsection{Verificações Numéricas}
\begin{equation}
\text{Verificar: } |S_{11}|^2 + |S_{21}|^2 \leq 1
\end{equation}

\section{Aplicações Práticas}

\subsection{Design de Filtros}
Os parâmetros S permitem:
\begin{itemize}
    \item Otimização de frequências de ressonância
    \item Controle de largura de banda
    \item Ajuste de acoplamento entre cavidades
\end{itemize}

\subsection{Análise de Antenas}
\begin{itemize}
    \item Cálculo de impedância de entrada
    \item Análise de casamento de impedância
    \item Determinação de eficiência de radiação
\end{itemize}

\section{Conclusões}

A teoria apresentada fornece uma base sólida para análise de guias cilíndricos. A implementação computacional dos parâmetros S permite:

\begin{itemize}
    \item Análise precisa de reflexão e transmissão
    \item Design otimizado de componentes de microondas
    \item Validação experimental de modelos teóricos
\end{itemize}

\section{Referências}

\begin{enumerate}
    \item Pozar, D. M. \textit{Microwave Engineering}, 4th Edition, Wiley, 2012.
    \item Collin, R. E. \textit{Foundations for Microwave Engineering}, 2nd Edition, McGraw-Hill, 1992.
    \item Ramo, S., Whinnery, J. R., Van Duzer, T. \textit{Fields and Waves in Communication Electronics}, Wiley, 1994.
    \item Harrington, R. F. \textit{Time-Harmonic Electromagnetic Fields}, McGraw-Hill, 1961.
\end{enumerate}

\appendix

\section{Tabela de Zeros das Funções de Bessel}

\begin{table}[H]
\centering
\caption{Primeiros zeros $p_{nm}$ de $J_n(x) = 0$}
\begin{tabular}{cccc}
\toprule
$n$ & $m=1$ & $m=2$ & $m=3$ \\
\midrule
0 & 2.405 & 5.520 & 8.654 \\
1 & 3.832 & 7.016 & 10.174 \\
2 & 5.135 & 8.417 & 11.620 \\
\bottomrule
\end{tabular}
\end{table}

\begin{table}[H]
\centering
\caption{Primeiros zeros $p'_{nm}$ de $J'_n(x) = 0$}
\begin{tabular}{cccc}
\toprule
$n$ & $m=1$ & $m=2$ & $m=3$ \\
\midrule
0 & 1.841 & 7.016 & 10.174 \\
1 & 3.832 & 5.331 & 8.536 \\
2 & 3.054 & 6.706 & 9.970 \\
\bottomrule
\end{tabular}
\end{table}

\section{Código de Implementação}

O código para implementação prática dos conceitos apresentados está disponível nos modelos Python do simulador, especificamente nos arquivos:
\begin{itemize}
    \item \texttt{Cilindrico\_model.py}: Implementação dos modos cilíndricos
    \item \texttt{Scattering\_model.py}: Cálculo da matriz de espalhamento
\end{itemize}

\end{document}